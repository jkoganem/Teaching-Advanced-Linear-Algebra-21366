\documentclass[reqno]{amsart}

\usepackage{amssymb,amsfonts,amsmath,amsthm}
\usepackage{hyperref}
\usepackage[left=.75in, right=.75in, top=.75in, bottom=.75in]{geometry}

% You can define custom commands and environments to suit your needs
% Text editors such as VS Code also support keyboard shortcuts, see https://code.visualstudio.com/docs/getstarted/keybindings
\newcommand{\R}{\mathbb{R}}

\newtheorem{thm}{Theorem}[section]
\newtheorem{cor}[thm]{Corollary}
\newtheorem{prop}[thm]{Proposition}
\newtheorem{lem}[thm]{Lemma}
\newtheorem{defn}[thm]{Definition}
\newtheorem{exmp}[thm]{Example}
\newtheorem{rem}[thm]{Remark}


\title{Paper title}

\author{Author 1}
\address{
Carnegie Mellon University\\
Pittsburgh, PA 15213, USA
}
\email[Author 1]{author1@andrew.cmu.edu}

\author{Author 2}
\address{
Carnegie Mellon University\\
Pittsburgh, PA 15213, USA
}
\email[Author 2]{author2@andrew.cmu.edu}

\keywords{Linear algebra}


\begin{document}

\begin{abstract}
Here is an abstract.
\end{abstract}

\maketitle

\tableofcontents

\section{Introduction}
\subsection{Main problem}
\subsection{Previous Work}
\subsection{Main results and discussion}
\subsection{Notation and outline}

\section{Let's prove some stuff}

\section{Applications}

\appendix
\section{Appendix for technical details}

\section*{Acknowledgements}

\nocite{*}
\bibliographystyle{abbrv}
\bibliography{bib.bib}

\end{document}
